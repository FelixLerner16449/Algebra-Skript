%This template was written by Elias Cleusters, Paul Oberdörster and Felix Lerner. This template is constantly getting updated so if you have any suggestions please let us know (name.surname@rwth-aachen.de).
% PREAMBLE
% =========

%Allgemein
%\usepackage[ngerman]{babel}                %!!!! removed should be declared in the main file
\usepackage[utf8]{inputenc}					% Input encoding
\usepackage[T1]{fontenc}					% T1 fonts 
%\usepackage{amsfonts}						% Mathe font
\usepackage{newcent}						% New Century Schoolbook
%\usepackage{fouriernc}						% Mathe font für newcent
\usepackage{dsfont}							% Symbole für Zahlbereiche
\usepackage{zi4}							% Font für typewriter style
\usepackage{makeidx}                        % z.B. Für ein Glossar am Ende.
\makeindex
\usepackage[most]{tcolorbox}                % z.B. für Enviroments, colorboxes halt
\usepackage{varwidth}                       %? um verschiedene Elemente besser nebeneinander alignen zu können, z.B. Bilder, diagramme etc?

%biblatex                                   % Für Referenzen/ Bibliographie/ Literaturverzeichnis. Die folgenden Zeilen gehören dazu
\usepackage[backend=biber, natbib=true, style=alphabetic, url=false, doi=true, eprint=false, giveninits=true]{biblatex}
\usepackage{csquotes}                               %such that biblatex doesnt cry
                                                    %\addbibresource{...} should be executed in main

\usepackage{ifthen}                                 % basically Ternary Operator
\usepackage{letltxmacro}							% Makros neu definieren mit mehr Optionen. Verstehe ich nicht, scheint aber nicht zu schaden.
\usepackage{booktabs}								% Qualitativ hochwertige horizontale Striche
\usepackage{wrapfig}								% Text um Abbildungen herumwrappen
\usepackage{microtype}								% Verbesserte Zeilenumbrüche. Scheint ganz gut zu sein und das generelle Textbild zu verbessern
\usepackage{calc}                                   % Basic arithmetic, die der Compiler auswerten kann
\usepackage{xcolor}									%Definition von Farben
\usepackage{graphicx}								%Einfügen von externen Grafiken
\usepackage{tikz}									%Erstellen von Grafiken

\usepackage[shortlabels]{enumitem}					%Verbesserte Aufzählungen
\setitemize			{leftmargin=*}
\setenumerate		{leftmargin=*}

%Allgemeine Layouteinstellungen
	
\renewcommand{\footnoterule}{						
	\noindent\rule{0.4\textwidth}{.4pt}
	\vspace{0.6em}}
\deffootnote{1em}{1em}								
	{\thefootnotemark\hspace{0.5em}}
	
\usepackage{chngcntr}
\counterwithout{footnote}{section}

\usepackage{scrlayer-scrpage}
\setkomafont{pageheadfoot}
	{\normalfont\itshape}
\pagestyle{scrheadings}		
\renewcommand{\sectionmark}[1]
	{\markboth{\thesection~#1~}{}}
\renewcommand{\sectionmark}[1]
	{\markright{\thesubsection\autodot~#1}}

\usepackage[titles]{tocloft}	
\renewcommand{\cftchapfont}{\normalfont\rmfamily\bfseries}            % TOC Font für Chapters

\setkomafont{chapter}{\normalfont\Huge\bfseries}
\setkomafont{section}{\normalfont\Large\bfseries}
\setkomafont{subsection}{\normalfont\large\bfseries}
\setkomafont{subsubsection}{\normalfont\bfseries}


				
\clubpenalty 					= 10000					
\widowpenalty 				= 10000					
\displaywidowpenalty 	= 10000					

\makeatletter
\@beginparpenalty			=	10000					
\makeatother

\definecolor{linkblue}{RGB}{0,0,0}
\definecolor{citeblue}{RGB}{0,0,0}

\usepackage[pdfstartpage= 1, pdfstartview= FitB, colorlinks= true, bookmarks=true, bookmarksopen=true, bookmarksnumbered=true, linkcolor=linkblue, citecolor=citeblue, pdfdisplaydoctitle=true]{hyperref}          %Referenzen sind hyperrefs.
\usepackage{cleveref}                           %allowing the format of references to be determined automatically according to the type of reference

% Mathematische Pakete
\usepackage{amssymb}
\usepackage{amsmath}
\usepackage{amsthm}
\usepackage{latexsym}
%\usepackage{stmaryrd}
\usepackage{esint}									\usepackage{faktor}
\usepackage{mathtools}								
\usepackage{polynom}								
\usepackage{cancel}									
\usepackage[all]{xy}            %xy matrix for diagramms
\usepackage{tensor}             %for Transformation matrices